% -*-latex-*-
\begin{nusmvCommand} {execute\_partial\_traces} {Executes partial traces on the model FSM}

\cmdLine{execute\_partial\_traces [-h] [-v] [-r] [-m | -o output-file]
    {-e engine [-a | trace\_number]}}

Executes traces stored in the Trace Manager.  If no trace is
specified, last registered trace is executed. Traces are not required
to be complete.  Upon succesful termination, a new complete trace is
registered in the Trace Manager.

\begin{cmdOpt}
\opt{-v} { Verbosely prints traces execution steps.}

\opt{-a}{ Prints all the currently stored traces.}

\opt{-r}{ Performs restart on complete states. When a complete state
  (i.e. a state which is non-ambiguosly determined by a complete
  assignment to state variables) is encountered, the re-execution
  algorithm is re-initialized, thus reducing computation time.}

\opt{-m}{ Pipes the output through the program specified by the
\shellvar{PAGER} shell variable if defined, else through the \unix
command \shellcommand{more}.}

\opt{-o \parameter{\filename{output-file}}}{Writes the output generated by the command to
\filename{output-file}.}

\opt{-e \parameter{\filename{engine}}}{Selects an engine for trace
  re-execution. It must be one of 'bdd', 'sat' or 'smt'.}

\opt{\natnum{trace\_number}}{ The (ordinal) identifier number of the trace to
 be printed. This must be the last argument of the command. Omitting
 the trace number causes the most recently generated trace to be executed.}
\end{cmdOpt}

\end{nusmvCommand}
