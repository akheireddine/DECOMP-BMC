% -*-latex-*-
\begin{nusmvCommand}{check\_pslspec} {Performs bdd-based PSL model checking}
\todo{AI: upper case for "bdd" in the label}

\cmdLine{check\_pslspec [-h] [-m | -o output-file] [-n number | -p
    \linebreak "\pslexpr [IN context]" | -P "name"]}

Check psl properties using bdd-based model checking.

A \pslexpr to be checked can be specified at command line using option
\commandopt{p}. Alternatively, option \commandopt{n} can be used for
checking a particular formula in the property database. If neither
\commandopt{n} nor \commandopt{p} are used, all the PSLSPEC formulas
in the database are checked.

See variable \varName{use\_coi\_size\_sorting} for changing properties
verification order.

\begin{cmdOpt}
\opt{-m}{ Pipes the output generated by the command in processing
      \code{PSLSPEC}{s} to the program specified by the \envvar{PAGER}
      shell variable if defined, else through the \unix command
      \shellcommand{more}.}

\opt{-o \parameter{\filename{\it output-file}}}{Writes the output
      generated by the command in processing \code{PSLSPEC}s to the file
      \filename{\it output-file}}

\opt{-p \parameter{"\pslexpr\newline\hspace*{6mm} [IN context]"}}{ A PSL formula to be
      checked.  \code{context} is the module instance name which the
      variables in \pslexpr must be evaluated in.}

\opt{-n \parameter{\natnum{number}}}{ Checks the PSL property with
      index \natnum{number} in the property database.}

\opt{-P \parameter{\natnum{name}}}{Checks the PSL property named \natnum{name} in
the property database.}

\end{cmdOpt}

\end{nusmvCommand}



